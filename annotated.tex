\documentclass[a4paper,11pt]{article}
\usepackage{beamerarticle, pgf, float}

%\usepackage{tikz}
%\usetikzlibrary{fadings}

\setjobnamebeamerversion{slides}

\newcommand{\ass}[2]{A^{e}_{#1}(#2)}
\newcommand{\vass}[1]{\boldsymbol{A}^{e}(#1)}
\newcommand{\lia}[1]{\ell_{#1}}
\newcommand{\val}[2]{V_{#1}(#2)}
\newcommand{\equ}[2]{E_{#1}(#2)}

\newcommand{\vv}{\boldsymbol{v}}

\newcommand{\sA}{\textup{A}}
\newcommand{\sB}{\textup{B}}

\begin{document}
  \begin{figure}[H]
    \begin{center}
      \includeslide[width=\textwidth]{titlepage}
    \end{center}
    \ttfamily
    Welcome to my talk about financial contagion.

    In fact I am suspicious of this terminology,
    which is why I have put those words in inverted commas.
    I am going to propose a model that owes more to pricing theory than to epidemiology.
  \end{figure}
  \begin{figure}[H]
    \begin{center}
      \includeslide[width=\textwidth]{baselquote}
    \end{center}
    \ttfamily
    The first of my two motivating quotes is from the Bank for International Settlements.
    The key point here is the estimate that roughly two thirds of losses attributed to conterparty credit risk
    during the financial crisis were due to fear of default rather than to actual default.

    The technology to account for the danger of default in a single counterparty is not new.
    The so-called structural model of credit risk dates back to the 1970s and is associated with
    names such as Merton, Black and Cox.
  \end{figure}
  \begin{figure}[H]
    \begin{center}
      \includeslide[width=\textwidth]{boequote}
    \end{center}
    \ttfamily
    My second quote is from the Bank of England. It recognises the need to account for this effect and it has a model
    (updated in 2019's `Forward-looking solvency contagion').

    As regulator, it must concern itself with system-wide risks and it has adopted the language of contagion.
    One thing affects another which affects another. It sounds like a fixed point argument and it is.
    This is the approach traditionally associated with network pricing, with papers by Eisenberg and Noe in 2001 and Rogers and Veraart in 2013.
    But they were modelling the impact of actual default.
    The Bank of England is applying the same logic, but to banks whose credit rating is merely impaired.

    The conclusion of the Bank of England is that there is nothing to worry about.
    I don't have its data so I can't take a view on that.
    I am also not going to criticise its assumptions. They may not be true but they are useful. I adopt them myself.
    But I see some issues with the model it derives from those assumptions.
  \end{figure}
  \begin{figure}[H]
    \begin{center}
      \includeslide[width=\textwidth]{outline}
    \end{center}
    \ttfamily
    My aim today is to share my intuition for the way I think this should be handled.
    So to start off with at least, I confine myself to a very simple case, that of two banks and zero recovery rate.
    Two banks because two dimensions is as many as I can comfortably fit on a slide.
    Zero recovery rate meaning that when a bank defaults, its creditors get nothing.
    That as well is to make the plots easier.
    Neither of these assumptions is required by the model itself as we will see nearer the end.
  \end{figure}
  \begin{figure}[H]
    \begin{center}
      \includeslide[width=\textwidth]{counterpartyfailure<1>}
    \end{center}
    \ttfamily
    The real economy I assume to evolve independently of the financial sector.
    It communicates with the financial sector through the vector $\vass{t}$.
    Component $\ass{j}{t}$ is the external assets value of bank $j$ and component $\ass{k}{t}$ is the external assets value of bank $k$.
    So $j$ might be Barclays and $k$ might be HSBC. These figures are in billions of pounds.

    This blue squiggle is a sample path, or simulation, over a year. I assume a correlated Geometric Brownian Motion.
    It is perhaps a big assumption but it is the assumption that the Bank of England also makes.
    Or weaker even, since the Bank of England dispenses with correlation.

    I have written $\lia{j}$ for the total liabilities of bank $j$
    (in BoE notation it is $L^{e}_{j}+\sum_{j}L_{jk}$).
    So if at any point $\ass{j}{t}$ were to fall below $\lia{j}$ then bank $j$ would fail.
    Graphically it would be if the path were to cross the thin grey line.
    This particular path didn't, but if I were to generate, say, thirty thousand such paths then some fraction of them would do so within the year.
    That fraction we call the probability of default at $\vass{0}$.
  \end{figure}
  \begin{figure}[H]
    \begin{center}
      \includeslide[width=\textwidth]{counterpartyfailure<2>}
    \end{center}
    \ttfamily
    I have done exactly that, and calculated probability of default from many different starting points.
    This contour plot shows survival probability, that is, one minus probability of default.
    It is the fair value per unit face value of bank $j$'s debt with one year to maturity,
    assuming an interest rate of zero.

    Note that the valuation $\val{j}{0}$ falls smoothly to zero at the failure boundary,
    where probability of default is one. Naturally these contour lines are absolutely vertical,
    since we have not yet introduced dependence on $\ass{k}{t}$.
  \end{figure}
  \begin{figure}[H]
    \begin{center}
      \includeslide[width=\textwidth]{counterpartyfailure<3>}
    \end{center}
    \ttfamily
    Here I do the same for bank $k$ with the exact same axes and same scale.
    Of course $\lia{k}$ stands for the total liabilities of bank $k$.
    Bank $k$ is assumed to fail, with its debt worthless, if $\ass{k}{t}$ falls below $\lia{k}$
    at any point during the year.

    I have described a Monte Carlo pricing algorithm that technically is justified by the hedging argument
    and no-arbitrage principles introduced by Merton, Black and Scholes in the 1970s.
    However they tended to use the language of partial differential equations,
    and confined themselves to one dimension for the underlying.
    But the same argument extends perfectly well to two dimensions or more.
  \end{figure}
  \begin{figure}[H]
    \begin{center}
      \includeslide[width=\textwidth]{balancesheets<1>}
    \end{center}
    \ttfamily
    The situation viewed through balance sheets.
    $\equ{j}{t}$ is the difference between $\ass{j}{t}$ and $\lia{j}$.
    If it is positive we call it equity.
    If it is not then bank $j$ has failed and is out of the game.
    This is an important point.
    In my first attempt at this I made a mistake.
    I inadvertently assumed that a bank would come back to life if its assets recovered.
    Not only was it a bad assumption but it made the maths harder.
  \end{figure}
  \begin{figure}[H]
    \begin{center}
      \includeslide[width=\textwidth]{balancesheets<2>}
    \end{center}
    \ttfamily
    It is time to introduce interbank exposures.
    These entries in green could represent many kinds of exposure, but it is most straightforward simply to imagine debt.
    Bank $j$ owes the amount $L_{jk}$ to bank $k$, payable after a year.
    The liability will not change for the period, whatever happens to asset prices, whatever happens to $k$.
    So I can just absorb it into $\lia{j}$ representing fixed total liabilities.

    On the other hand where it reappears on the asset side of bank $k$'s balance sheet,
    it is mediated by $\val{j}{t}$, a function of time and (external assets) space, and is counted as an additional asset.
    (For ease of comparison I have adopted BoE notation wherever possible
    but its stochastic flavour may obscure some functional dependencies.)

    Likewise for $L_{kj}$ the amount owed by bank $k$ to bank $j$.
    Because they now have these additional assets, the external assets of each bank can fall further before it fails.
  \end{figure}
  \begin{figure}[H]
    \begin{center}
      \includeslide[width=\textwidth]{interbankexposures<1>}
    \end{center}
    \ttfamily
    In this plot we can see that the failure boundary for bank $j$ has shifted to the left, at least in the region where bank $k$ survives.
    The contour lines all move in sympathy.
    If bank $k$ does fail, if $\ass{k}{t}$ falls below the thin grey line (even if it subsequently recovers), then we are back to our original problem.
    The vertical lines are unchanged from the earlier slides.
    So here we have two functions plotted on the same axes, one for both banks surviving and one for just bank $j$ surviving.

    This suggests the need explicitly to track the set of surviving banks.
    It is a third argument to our debt valuation function, after time and external assets value.
  \end{figure}
  \begin{figure}[H]
    \begin{center}
      \includeslide[width=\textwidth]{interbankexposures<2>}
    \end{center}
    \ttfamily
    The analogous plot for bank $k$. So long as asset prices stay to the right of the thin grey line,
    that is, so long as bank $j$ remains solvent,
    the failure boundary for bank $k$ is the lower one.
    But if the asset path crosses that line and bank $j$ fails, then bank $k$'s failure boundary reverts to the upper one, as in the earlier slide.
    Note that the two solutions are continuous with one another where they meet at the grey line.
    This is a natural consequence of our algorithm. It is a property of random walks that,
    as you approach a boundary, the probability of crossing it rises to one.
    So there can be no sudden jumps in expectation.
  \end{figure}
  \begin{figure}[H]
    \begin{center}
      \includeslide[width=\textwidth]{comparison<1>}
    \end{center}
    \ttfamily
    Although the Bank of England's data are not public, its algorithm is.
    It has been coded up in Python and uploaded to GitHub.
    This plot shows the result when it is run against the estimated figures that we have been using.
    My own solution is reproduced in fainter lines for comparison.

    There is a large region where the Bank of England model says everything is fine, but where my model indicates failure for both banks.
    At the edge of that region, the Bank of England's model's results display a very sudden discontinuity,
    where the debt of both banks goes from close to 100\% of face value to worthless in no distance at all.
  \end{figure}
  \begin{figure}[H]
    \begin{center}
      \includeslide[width=\textwidth]{comparison<2>}
    \end{center}
    \ttfamily
    The discontinuity appears in the same place when pricing bank $k$'s debt.
    These discontinuities appear because the Bank of England is not, like us, solving the two-dimensional problem directly.
    Instead it applies a fixed point procedure to force together two copies of the solution to the one-dimensional problem,
    which has an analytic expression.
    For some sets of input, like this one, the procedure will have multiple solutions.
    The program simply selects the most optimistic one.
    So what we see here is something akin to hysteresis.

    There is no danger of multiple solutions in the approach I am describing today.
    
    Our solutions do match at either extreme because I have kept as close as possible to the assumptions of the Bank of England.
  \end{figure}
  \definecolor{lightblue}{rgb}{0.7,0.7,1.0}
  \begin{figure}[H]
    \begin{center}
      %\begin{tikzpicture}
        %\path [draw=blue, fill=lightblue, path fading=north]
        %(0.039, 0.36) rectangle (0.223, 1.20);
        \includeslide[width=\textwidth]{maturityprofiles}
      %\end{tikzpicture}
    \end{center}
    \ttfamily
    In particular I have kept to the assumption that all interbank debt has exactly twelve months to maturity,
    and that it somehow stays out at twelve months to maturity.
    I am replacing this now with my own arbitrary assumption,
    that when binned according to time to maturity, the amount of debt falls off smoothly.

    Both assumptions have the same average maturity of one year.
    My assumption has the virtue that it makes calculations easier when we pass to differential equations.
  \end{figure}
  \begin{figure}[H]
    \begin{center}
      \includeslide[width=\textwidth]{revisedprofile<1>}
    \end{center}
    \ttfamily
    The revised plot looks qualitatively like the earlier one.
  \end{figure}
  \begin{figure}[H]
    \begin{center}
      \includeslide[width=\textwidth]{revisedprofile<2>}
    \end{center}
    \ttfamily
    You may be thinking that I have not addressed the question of how to find the failure boundary in the first place.
    It would seem that we need to know the debt valuation function to find the boundary
    and that we need to know the boundary to solve for the debt valuation function.
    But we are saved from having to resolve this with some sort of fixed point analysis by that property of continuity.

    At the failure boundary itself, wherever it may be, we know that the solution for both banks surviving is continuous with the solution for just one bank surviving.
    Bank $k$ in this plot.
    So we can use the solution for just one bank surviving as a proxy for the solution for both banks surviving
    when it comes to calculating the exact location of the boundary.
  \end{figure}
  \begin{figure}[H]
    \begin{center}
      \includeslide[width=\textwidth]{dependencystructure}
    \end{center}
    \ttfamily
    That is, I am describing a computation with the structure represented on the left here.
    In order to compute the solutions for both banks surviving,
    I need boundary conditions supplied by the two solutions for just one bank surviving.
    One might also say that each of those two solutions depends on conditions supplied by the solution for no banks surviving, although that is trivial.
    Therefore the algorithm must proceed upwards from the bottom of this graph to the top,
    from the solution associated with the set of no banks to the solution associated with the set of both banks,
    just as I structured this talk.

    These ideas extend in a straightforward way to a larger number of banks.
    The number of solutions required rises exponentially and the dimension of each solution also rises with the number of banks.
    Nevertheless I have found it possible to perform this exercise for a system of seven banks, the number assumed major players in the UK sector.
    That program solved differential equations rather than running Monte Carlo, because Monte Carlo is not good for calculating boundaries.
  \end{figure}
  \begin{figure}[H]
    \begin{center}
      \includeslide[width=\textwidth]{supportingtheory}
    \end{center}
    \ttfamily
    There is a theory behind all this, and here is a taste of it.
    This theorem tells us that if one set of banks is a subset of another
    then the debt valuation function associated with the first takes lower values everywhere than the debt valuation function associated with the second.
    Each boldface $\vv$ is vector-valued and the inequality applies to every component.
    It is not a surprising result, since it is hard to see how an increased number of surviving banks could damage the health of any party.

    The corollary, following trivially, is what is really useful.
    If every bank in a set $\sA$ is solvent according to the solution associated with some some subset of $\sA$
    then every bank in $\sA$ is solvent according to the solution associated with $\sA$ itself.
    Together with the principle of continuity mentioned earlier,
    this is what allows us to use solutions associated with subsets of $\sA$ to determine
    the boundary conditions we need to set up the problem for $\sA$ itself.
  \end{figure}
  \begin{figure}[H]
    \begin{center}
      \includeslide[width=\textwidth]{recoveryrate<1>}
    \end{center}
    \ttfamily
    We have come to my last couple of slides.
    Nonzero recovery rates have been a recurrent theme in the literature.
    Eisenberg and Noe had in effect a recovery rate of one at the point of failure.
    Rogers and Veraart generalised that to values of $\beta$ between zero and one.

    For this plot I have adopted a Rogers and Veraart-type model to value the debt of banks in default,
    while using the procedure I have described for banks still surviving.
    The straight lines and sharp corners we see left of and below the failure boundaries
    are a consequence of the Rogers and Veraart fixed point logic applied to banks in default.
    The smoother contours above and to the right come from
    the solution to the diffusion equation for surviving banks that this talk has been about.
  \end{figure}
  \begin{figure}[H]
    \begin{center}
      \includeslide[width=\textwidth]{recoveryrate<2>}
    \end{center}
    \ttfamily
    In summary.

    I propose a model for the valuation of interbank debt in a network,
    and hence for the assessment of costs associated with financial contagion.
    The problem and the assumptions are not new,
    but as far as I am aware I have arrived at a new methodology.
    Results suggest that if one is to take my analysis seriously then the
    Bank of England's model is overoptimistic in a large region of its domain.
  \end{figure}
\end{document}